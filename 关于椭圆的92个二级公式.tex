% !TEX encoding = UTF-8 Unicode
\documentclass{article}
\usepackage[margin=2.5cm]{geometry} % 设置边距为 2.5cm
\usepackage{amsmath}
\everymath{\displaystyle}
\everydisplay{\displaystyle}

\usepackage{fancyhdr} % 页脚设置
\usepackage{lastpage}

\pagestyle{fancy}
\fancyfoot[C]{第\thepage 页 共\pageref{LastPage}页}

\usepackage{multicol} %用于实现在同一页中实现不同的分栏

\usepackage[UTF8]{ctex}

\usepackage{datetime} % 使用 datetime 宏包来格式化日期

\usepackage{hyperref}
\hypersetup{
colorlinks=true,
linkcolor=blue,
urlcolor=blue,
citecolor=blue
} 
\usepackage{tocloft} % 加载宏包

\usepackage{enumitem} % 使用enumitem包来自定义列表项
\newcounter{myitemcounter} % 创建一个名为myitemcounter的计数器
\setcounter{myitemcounter}{0} % 初始化计数器的值为0


\title{关于椭圆的92个二级公式}
\author{CjB}
\newdateformat{chinesedate}{\THEYEAR 年\THEMONTH 月\THEDAY 日}
\date{\chinesedate\today}

\setlength{\parindent}{0pt} % 取消段落缩进



\begin{document}
\maketitle % 生成标题

\tableofcontents % 生成目录
%%%%%%%%%%%%%%%%%%%%%%%%%%%%%%%%%%%%%%%%%%%%%%%%%%%%%%%%%%%%%%%%%%%%%%%
\newpage

\section{\href{https://www.bilibili.com/video/BV12h41147FS/?spm_id_from=333.999.top_right_bar_window_history.content.click&vd_source=9b93cf0c44112a0e78e9ab65807c27c8}{关于椭圆的92个二级公式}}

\begin{enumerate}[label=\arabic*.] % 使用enumitem包来定义自定义标签,\arabic*表示使用阿拉伯数字并自动编号

\begin{multicols}{3} % 分三栏 若花括号中为3则是分三列

\item $\left| PF_1 \right| + \left| PF_2 \right| = 2a$.

\item 标准方程$\frac{x^2}{a^2} + \frac{y^2}{b^2} = 1$.

\item $\displaystyle\frac{PF_1}{d_1}=e < 1$.

\end{multicols}

\item 点$P$处的切线$PT$平分$\triangle PF_1F_2$在点$P$处的外角.

\item $PT$平分$\triangle PF_1F_2$在点$P$处的外角,则焦点在直线$PT$上的射影$H$点的轨迹是以长轴为直径的圆,除去长轴的两个端点.

\item 以焦点弦$PQ$为直径的圆必与对应准线相离.

\item 以焦点半径$PF_1$为直径的圆必与以长轴为直径的圆内切.

\item 设$A_1$、$A_2$为椭圆的左右顶点,则$\triangle PF_1F_2$在边$PF_2$(或$PF_1$)上的旁切圆,必与$A_1A_2$所在的直线切于$A_1$(或$A_2$).

\item 设椭圆$\frac{x^2}{a^2} + \frac{y^2}{b^2} = 1 \left(a > b > 0\right)$的两个顶点为$A_1\left(-a,0\right)$,$A_2\left(a,0\right)$,与$y$轴平行的直线交椭圆于点$P_1$、$P_2$时,$A_1P_1$与$A_2P_2$交点的轨迹方程是$\frac{x^2}{a^2} - \frac{y^2}{b^2} = 1 $. 

\item 若$P_0\left(x_0,y_0\right)$在椭圆$\displaystyle\frac{x^2}{a^2} + \frac{y^2}{b^2} = 1$上,则过$P_0$的椭圆的切线方程是$\frac{x_0x}{a^2} + \frac{y_0y}{b^2} = 1$.

\item 若$P_0\left(x_0,y_0\right)$在椭圆$\displaystyle\frac{x^2}{a^2} + \frac{y^2}{b^2} = 1$外,则过$P_0$作椭圆的两条切线切点分别为$P_1$、$P_2$,则切点弦$P_1P_2$的直线方程是$\displaystyle\frac{x_0x}{a^2} + \frac{y_0y}{b^2} = 1$.

\item $AB$是椭圆$\displaystyle\frac{x^2}{a^2} + \frac{y^2}{b^2} = 1 $的不平行于对称轴的弦,$M$为$AB$的中点,则$k_{OM} \cdot k_{AB} = -\frac{b^2}{a^2}$. 

\item 若$P_0\left(x_0,y_0\right)$在椭圆$\displaystyle\frac{x^2}{a^2} + \frac{y^2}{b^2} = 1$内,则被$P_0\left(x_0,y_0\right)$所平分的中点弦的方程是$\frac{x_0x}{a^2} + \frac{y_0y}{b^2} = \frac{x_0^2}{a^2} + \frac{y_0^2}{b^2}$.

\item 若$P_0\left(x_0,y_0\right)$在椭圆$\displaystyle\frac{x^2}{a^2} + \frac{y^2}{b^2} = 1$内,则过$P_0\left(x_0,y_0\right)$的弦的中点的轨迹方程是$\frac{x^2}{a^2} + \frac{y^2}{b^2} = \frac{x_0x}{a^2} + \frac{y_0y}{b^2}$.

\item 若$PQ$是椭圆$\frac{x^2}{a^2} + \frac{y^2}{b^2} = 1 \left(a > b > 0\right)$上对于中心张直角的弦,则$\frac{1}{r_1} + \frac{1}{r_2} = \frac{1}{a^2} + \frac{1}{b^2}\left(r_1 = \left| OP \right|, r_2 = \left| OQ \right| \right)$. 

\item 若椭圆$\frac{x^2}{a^2} + \frac{y^2}{b^2} = 1 \left(a > b > 0\right)$上中心张直角的弦$L$所在的直线方程为$Ax+By=1\left(AB\neq0\right)$,则
\begin{multicols}{2} % 分两栏 若花括号中为3则是分三列
(1)$\frac{1}{a^2} + \frac{1}{b^2} = A^2 + B^2$

(2)$L=\frac{2\sqrt{a^4A^2+b^4B^2}}{a^2A^2+b^2B^2}$. 
\end{multicols}

\item 给定椭圆$C_1:b^2x^2+a^2y^2 = a^2b^2\left(a > b > 0\right)$,$C_2:b^2x^2+a^2y^2 = \left(\frac{a^2-b^2}{a^2+b^2}ab\right)^2$,则

\item[(i)] 对$C_1$上任意给定的点$P\left(x_0,y_0\right)$,它的任一直角弦必须经过$C_2$上任一定点$M\left(\frac{a^2-b^2}{a^2+b^2}x_0,-\frac{a^2-b^2}{a^2+b^2}y_0\right)$, 

\item[(ii)] 对$C_2$上任意给定的点$P'\left(x_0',y_0'\right)$在$C_1$上存在唯一的点$M'$,使得点$M'$的任意直角弦都经过点$P'$.

\item 设$P\left(x_0, y_0\right)$为椭圆(或圆)$C: \frac{x^2}{a^2} + \frac{y^2}{b^2} = 1\left(a > 0, b > 0\right)$上的一点,$P_1P_2$为曲线$C$的动弦,且弦$PP_1$的斜率$k_1$与$PP_2$的斜率$k_2$都存在,则直线$P_1P_2$通过定点$M\left(mx_0,-my_0\right)\left(m \neq 1\right)$的充要条件是$$k_1 \cdot k_2 = -\frac{1+m}{1-m}\cdot \frac{b^2}{a^2}$$.

\item 过椭圆$\frac{x^2}{a^2} + \frac{y^2}{b^2} = 1\left(a > 0, b > 0\right)$上任一点$A\left(x_0,y_0\right)$任意作两条倾斜角互补的直线交椭圆于$B$、$C$两点,则直线$BC$有固定斜率$k_{BC}=\frac{b^2x_0}{a^2y_0}$. 

\item 设椭圆$b^2x^2+a^2y^2 = a^2b^2\left(a > b > 0\right)$的左右焦点分别为$F_1$,$F_2$,点$P$为椭圆上任意一点,满足$\angle F_1PF_2 = \gamma$,则椭圆的焦点三角形的面积为$S_{\triangle F_1PF_2} = b^2 \tan\frac{\gamma}{2}$,且$P$点的坐标为$\left(\pm \frac{a}{c} \sqrt{c^2-b^2\tan^2\frac{\gamma}{2}},\pm\frac{b^2}{c}\tan\frac{\gamma}{2}\right)$. 

\item 若$P$为椭圆$\frac{x^2}{a^2} + \frac{y^2}{b^2} = 1 \left(a > b > 0\right)$上异于长轴端点的任一点,$F_1$,$F_2$是焦点,$\triangle PF_1F_2 = \alpha$,$\triangle PF_2F_1 = \beta$,则$\frac{a-c}{a+c} = \tan{\frac{\alpha}{2}}\tan{\frac{\beta}{2}}$. 

\item 椭圆$\frac{x^2}{a^2} + \frac{y^2}{b^2} = 1 \left(a > b > 0\right)$的焦半径公式:$\left| MF_1 \right| = a + ex_0$,$\left| MF_2 \right| = a - ex_0$,$F_1\left(-c,0\right)$,$F_2\left(c,0\right)$,$M\left(x_0,y_0\right)$. 

\item 若椭圆$\frac{x^2}{a^2} + \frac{y^2}{b^2} = 1 \left(a > b > 0\right)$的左、右焦点分别为$F_1$,$F_2$,左准线为$L$,则当$\sqrt{2}-1<e<1$时,可在椭圆上求一点$P$,使得$PF_1$是$P$到对应准线距离$d$与$PF2$的比例中项.

\item $P$为椭圆$\frac{x^2}{a^2} + \frac{y^2}{b^2} = 1 \left(a > b > 0\right)$上任一点,$F_1$、$F_2$为两焦点,$A$为椭圆内一定点,则$2a-\left|AF_2 \right| \leq \left| PA \right| + \left| PF_1 \right| \leq 2a + \left| AF_2 \right|$,当且仅当$A$,$F_2$,$P$三点共线时,等号成立. 

\item 椭圆$\frac{x^2}{a^2} + \frac{y^2}{b^2} = 1 \left(a > b > 0\right)$上存在两点关于直线$l:y=k\left(x - x_0\right)$对称的充要条件是$$x_0^2 \leq \frac{(a^2-b^2)^2}{a^2+b^2k^2}$$.

\item 过椭圆焦半径的端点作椭圆的切线,与以长轴为直径的圆相交,则相应交点与相应焦点的连线必与切线垂直. 

\item 过椭圆焦半径的端点作椭圆的切线交相应准线于一点,则该点与焦点的连线必与焦半径互相垂直. 

\item $P$是椭圆$ 
\begin{cases}
    x=a\cos\varphi \\
    y=b\sin\varphi
\end{cases}(a>b>0)$上一点,则点$P$对椭圆两焦点张直角的充要条件是$e^2=\frac{1}{1+\sin^2\varphi}$. 

\item 设$A$,$B$为椭圆$\frac{x^2}{a^2} + \frac{y^2}{b^2} =k\left(k>0, k\neq 1\right)$上两点,其直线$AB$与椭圆$\frac{x^2}{a^2} + \frac{y^2}{b^2} = 1$相交于$P$、$Q$两点,则$AP=BQ$. 

\item 在椭圆$\frac{x^2}{a^2} + \frac{y^2}{b^2} = 1$中,定长为$2m\left(0<m\leq a\right)$的弦中点的轨迹方程为:

$m^2=\left[1-\left(\frac{x^2}{a^2} + \frac{y^2}{b^2}\right)\right]\left(a^2\cos^2\alpha+b^2\sin^2\alpha\right)$,其中$\tan\alpha=-\frac{bx}{ay}$,当$y=0$时,$\alpha=90^\circ$. 

\item 设$S$为椭圆$\frac{x^2}{a^2} + \frac{y^2}{b^2} = 1 \left(a > b > 0\right)$的通径,定长线段$L$的两端点$A$,$B$在椭圆上移动,记$\left| PF_1 AB\right|=l$,$M\left(x_0,y_0\right)$是$AB$的中点,则当$l \ge\Phi S$时,有$\left(x_0\right)_{max}=\frac{a^2}{c}-\frac{l}{2e}\left(c^2=a^2-b^2,e=\frac{c}{a}\right)$,当$l<\Phi S$时,有$\left(x_0\right)_{max}=\frac{a}{2b}\sqrt{4b^2-l^2},\left(x_0\right)_{min}=0$. 

\item 椭圆$\frac{x^2}{a^2} + \frac{y^2}{b^2} = 1$与直线$Ax+By+C=0$有公共点的充要条件是$A^2a^2+B^2b^2\ge C^2$. 

\item 椭圆$\frac{\left(x-x_0\right)^2}{a^2} + \frac{\left(y-y_0\right)^2}{b^2} = 1$与直线$Ax+By+C=0$有公共点的充要条件是
$$A^2a^2+B^2b^2\ge (Ax_0+By_0+C)^2$$. 

\item 设椭圆$\frac{x^2}{a^2} + \frac{y^2}{b^2} = 1 \left(a > b > 0\right)$的两个焦点为$F_1$,$F_2$,$P$(异于长轴端点)为椭圆上任意一点,在$\triangle PF_1F_2$中,记$\angle F_1PF_2 =\alpha$,$\angle PF_1F_2=\beta $,$\angle F_1F_2P = \gamma$,则有$\frac{\sin \alpha}{\sin\beta +\sin\gamma}=\frac{c}{a}=e$. 

\item 经过椭圆$\frac{x^2}{a^2} + \frac{y^2}{b^2} = 1 \left(a > b > 0\right)$的长轴的两端点$A_1$和$A_2$的切线,与椭圆上任一点的切线相交于$P_1$和$P_2$,则
$\left|P_1A_1 \right|\cdot\left|P_2A_2 \right|=b^2$. 

\item 已知椭圆$\frac{x^2}{a^2} + \frac{y^2}{b^2} = 1 \left(a > b > 0\right)$,$O$为坐标原点,$P$、$Q$为椭圆上两动点,且$OP\perp OQ$,则

(1)$\frac{1}{\left|OP \right|^2}+\frac{1}{\left|OQ \right|^2}=\frac{1}{a^2}+\frac{1}{b^2}$;

(2)$\left| OP\right|^2 + \left| OQ\right|^2 $的最小值是
$\frac{4a^2b^2}{a^2+b^2}$; 

(3)$S_{\triangle OPQ}$的最小值是$\frac{a^2b^2}{a^2+b^2}$. 

\item $MN$是经过椭圆$\frac{x^2}{a^2} + \frac{y^2}{b^2} = 1 \left(a > b > 0\right)$焦点的任一弦,若$AB$是经过椭圆中心$O$且平行于$MN$的弦,则$\left|AB\right|^2=2a\left|MN\right|$. 

\item $MN$是经过椭圆$\frac{x^2}{a^2} + \frac{y^2}{b^2} = 1 \left(a > b > 0\right)$焦点的任一弦,若过椭圆中心$O$的半弦$OP \perp MN$,则$$\frac{2}{a\left|MN\right|}+\frac{1}{\left|OP\right|^2} = \frac{1}{a^2} + \frac{1}{b^2}$$. 

\item 设椭圆$\frac{x^2}{a^2} + \frac{y^2}{b^2} = 1 \left(a > b > 0\right)$,$M\left(m,0\right)$或$\left(0,m\right)$为其对称轴上除中心,顶点外的任一点,过$M$引一条直线与椭圆相交于$P$、$Q$两点,则直线$A_1P$、$A_2Q$($A_1$、$A_2$为对称轴上的两顶点)的交点$N$在直线$l:x=\frac{a^2}{m}$或($y=\frac{b^2}{m}$)上. 

\item 设过椭圆焦点$F$作直线与椭圆相交$P$、$Q$两点,$A$为椭圆长轴上一个顶点,连结$AP$和$AQ$分别交相应于焦点$F$的椭圆准线于$M$、$N$两点,则$NF\perp MF$. 

\item 过椭圆一个焦点$F$的直线与椭圆交于$P$、$Q$两点,$A_1$、$A_2$为椭圆长轴上的顶点,$A_1P$和$A_2Q$交于点$M$,$A_2P$和$A_1Q$交于点$N$,则$MF \perp NF$. 

\item 设椭圆方程$\frac{x^2}{a^2} + \frac{y^2}{b^2} = 1$,则斜率为$k\left(k\neq 0\right)$的平行弦的中点必在直线$l:y=kx$的共轭直线$y=k'x$上,而且$kk'=-\frac{b^2}{a^2}$. 

\item 设$A$、$B$、$C$、$D$为椭圆$\frac{x^2}{a^2} + \frac{y^2}{b^2} = 1$上四点,$AB$、$BC$所在直线的倾斜角分别为$\alpha$、$\beta$,直线$AB$与$CD$相交于点$P$,且点$P$不在椭圆上,则$\frac{\left|PA\right|\cdot\left|PB\right|}{\left|PC\right|\cdot\left|PD\right|} = \frac{b^2\cos^2\beta+a^2\sin^2\beta}{b^2\cos^2\alpha+a^2\sin^2\alpha}$. 

\item 已知椭圆$\frac{x^2}{a^2} + \frac{y^2}{b^2} = 1 \left(a > b > 0\right)$,点$P$为其上一点,$F_1$、$F_2$为椭圆的焦点,$\angle F_1PF_2$的外(内)角平分线为$l$,作$F_1$、$F_2$分别垂直$l$于$R$、$S$,当$P$跑遍整个椭圆时,$R$、$S$形成的轨迹方程是$x^2+y^2 = a^2
\left(
c^2y^2 = \frac{\left[a^2y^2 + b^2x\left(x\pm c\right) \right]^2}{a^2y^2 + b^2\left(x\pm c\right)^2}
\right)$. 

\item 设$\triangle ABC$内接于椭圆$\Gamma$,且$AB$为$\Gamma$的直径,$l$为$AB$的共轭直径所在的直线,$1$分别交直线$AC$、$BC$于$E$和$F$,又$D$为$l$上一点,则$CD$与椭圆$\Gamma$相切的充要条件是$D$为$EF$的中点. 

\item 过椭圆$\frac{x^2}{a^2} + \frac{y^2}{b^2} = 1 \left(a > b > 0\right)$的右焦点$F$作直线交椭圆右支于$M$、$N$两点,弦$MN$的垂直平分线交$x$轴于$P$,则$\frac{\left|PF\right|}{\left|MN\right|}=\frac{e}{2}$. 

\item 设$A\left(x_1,y_1\right)$是椭圆$\frac{x^2}{a^2} + \frac{y^2}{b^2} = 1 \left(a > b > 0\right)$上任一点,过$A$作一条斜率为$-\frac{b^2x_1}{a^2y_1}$的直线$L$,又设$d$是原点到直线$L$的距离,$r_1$、$r_2$分别是$A$到椭圆两焦点的距离,则$\sqrt{r_1r_2}d=ab$. 

\item 已知椭圆$\frac{x^2}{a^2} + \frac{y^2}{b^2} = 1 \left(a > b > 0\right)$和$\frac{x^2}{a^2} + \frac{y^2}{b^2} = \lambda \left(0 < \lambda < 1\right)$,一直线顺次与它们相交于$A$、$B$、$C$、$D$四点,则$\left|AB \right| = \left|CD \right|$. 

\item 已知椭圆$\frac{x^2}{a^2} + \frac{y^2}{b^2} = 1 \left(a > b > 0\right)$,$A$、$B$是椭圆上的两点,线段$AB$的中垂线与$x$轴相交于点$P(x_0,0)$,则$-\frac{a^2-b^2}{a}<x_0<\frac{a^2-b^2}{a}$. 

\item 设$P$点是椭圆$\frac{x^2}{a^2} + \frac{y^2}{b^2} = 1 \left(a > b > 0\right)$上异于长轴端点的任一点,$F_1$、$F_2$为其焦点,记$\angle F_1PF_2=\theta$,则

(1)$\left|PF_1\right|\cdot \left|PF_2\right|=\frac{2b^2}{1+\cos\theta}$

(2)$S_{\triangle PF_1F_2} = b^2\tan\frac{\theta}{2}$

\item 设过椭圆的长轴上一点$B(m,0)$作直线与椭圆相交于$P$、$Q$两点,$A$为椭圆长轴的左顶点,连结$AP$和$AQ$分别交相应于过$H$点的直线$MN: x=n$于$M$,$N$两点,则$$\angle MBN=90^\circ \Leftrightarrow \frac{a-m}{a+m} = \frac{a^2\left(n-m\right)^2}{b^2\left(n+a\right)^2}$$. 

\item $L$是经过椭圆$\frac{x^2}{a^2} + \frac{y^2}{b^2} = 1 \left(a > b > 0\right)$的长轴顶点$A$且与长轴垂直的直线,$E$、$F$是椭圆两个焦点,$e$是离心率,点$P$在$L$上,若$\angle EPF=\alpha$,则$\alpha$是锐角且$\sin\alpha\leq e$或$\alpha\leq \arcsin e$(当且仅当$\left|PH\right|=b$时取等号). 

\item $L$是椭圆$\frac{x^2}{a^2} + \frac{y^2}{b^2} = 1 \left(a > b > 0\right)$的准线,$A$、$B$是椭圆的长轴两顶点,点$P$在$L$上,$e$是离心率,$\angle EPF=\alpha$,$H$是$L$与$x$轴的交点,$c$是半焦距,则$\alpha$是锐角且$\sin\alpha\leq e$或$\alpha\leq \arcsin e$(当且仅当$\left|PH\right|=\frac{ab}{c}$时取等号). 

\item $L$是椭圆$\frac{x^2}{a^2} + \frac{y^2}{b^2} = 1 \left(a > b > 0\right)$的准线,$E$、$F$是两个焦点,$H$是$L$与$x$轴的交点,点$P$在$L$上,$\angle EPF = \alpha$,$e$是离心率,$c$是半焦距,则$\alpha$为锐角且$\sin\alpha\leq e^2$或$\alpha\leq \arcsin e^2$(当且仅当$\left|PH\right|=\frac{b}{c}\sqrt{a^2+c^2}$时取等号). 

\item 已知椭圆$\frac{x^2}{a^2} + \frac{y^2}{b^2} = 1 \left(a > b > 0\right)$,直线$L$通过其右焦点$F_2$,且与椭圆相交于$A$、$B$两点,将$A$、$B$与椭圆左焦点$F_1$连结起来,则$b^2\leq \left|F_1A \right|\cdot\left|F_1B \right|\leq \frac{\left(2a^2-b^2\right)^2}{a^2}$(当且仅当$AB \perp x$轴时右边不等式取等号,当且仅当$A$、$F_1$、$B$三点共线时左边不等式取等号). 

\item 设$A$、$B$是椭圆$\frac{x^2}{a^2} + \frac{y^2}{b^2} = 1 \left(a > b > 0\right)$的长轴两端点,$P$是椭圆上一点,$\angle PAB = \alpha$,$\angle PBA = \beta$,$\angle BPA = \gamma$,$c$是椭圆的半焦距,$e$是椭圆的离心率,则有

(1)$\left|PA \right| = \frac{2ab^2\left|\cos\alpha\right|}{a^2-c^2\cos^2\alpha}$,
(2)$\tan\alpha\tan\beta=1-e^2$,
(3)$S_{\triangle PAB} = \frac{2a^2b^2}{b^2-a^2}\cot\gamma$.

\item 设$A$、$B$是椭圆$\frac{x^2}{a^2} + \frac{y^2}{b^2} = 1 \left(a > b > 0\right)$长轴上分别位于椭圆内(异于原点)、外部的两点,且$x$、$x$的横坐标$x_A\cdot x_B=a^2$,

(1)若过$A$点引直线与这椭圆相交于$P$、$Q$两点,则$\angle PBA=\angle QBA$;

(2)若过$B$引直线与这椭圆相交于$P$、$Q$两点,则$\angle PAB+\angle QAB=180^\circ$. 

\item 设$A$、$B$是椭圆$\frac{x^2}{a^2} + \frac{y^2}{b^2} = 1 \left(a > b > 0\right)$长轴上分别位于椭圆内(异于原点)、外部的两点,

(1)若过$A$点引直线与这椭圆相交于$P$、$Q$两点,(若$BP$交椭圆于两点,则$P$、$Q$不关于$x$轴对称),且$\angle PBA=\angle QBA$,则点$A$、$B$的横坐标$x_A$、$x_B$满足$x_A\cdot x_B=a^2$;

(2)若过$B$点引直线与这椭圆相交于$P$、$Q$两点,且$\angle PAB+\angle QAB=180^\circ$,则点$A$、$B$的横坐标满足$x_A\cdot x_B=a^2$. 

\item 设$A$, $A'$是椭圆$\frac{x^2}{a^2} + \frac{y^2}{b^2} = 1$的长轴的两个端点,$QQ'$是与$AA'$垂直的弦,则直线$AQ$与$A'Q'$的交点$P$的轨迹是双曲线$\frac{x^2}{a^2}-\frac{y^2}{b^2}=1$. 

\item 过椭圆$\frac{x^2}{a^2} + \frac{y^2}{b^2} = 1 \left(a > b > 0\right)$的左焦点$F$作互相垂直的两条弦$AB$、$CD$则$$\frac{8ab^2}{a^2+b^2}\leq\left|AB\right| + \left|CD\right|\leq\frac{2\left(a^2+b^2\right)}{a}$$. 

\item 到椭圆$\frac{x^2}{a^2} + \frac{y^2}{b^2} = 1 \left(a > b > 0\right)$两焦点的距离之比等于$\frac{a-c}{b}$($c$为半焦距)的动点$M$的轨迹是姐妹圆$$\left(x\pm a\right)^2+y^2=b^2$$
$e$为离心率. 

\item 到椭圆$\frac{x^2}{a^2} + \frac{y^2}{b^2} = 1 \left(a > b > 0\right)$的长轴两端点的距离之比等于$\frac{a-c}{b}$($c$为半焦距)的动点$M$的轨迹是姐妹圆$$\left(x\pm \frac{a}{e}\right)^2+y^2=\left(\frac{b}{e}\right)^2$$
$e$为离心率. 

\item 到椭圆$\frac{x^2}{a^2} + \frac{y^2}{b^2} = 1 \left(a > b > 0\right)$的两准线和$x$轴的交点的距离之比为$\frac{a-c}{b}$($c$为半焦距)的动点的轨迹是姐妹圆$$\left(x\pm \frac{a}{e^2}\right)^2+y^2=\left(\frac{b}{e^2}\right)^2$$
$e$为离心率. 

\item 已知$P$是椭圆$\frac{x^2}{a^2} + \frac{y^2}{b^2} = 1 \left(a > b > 0\right)$上一个动点,$A$、$A'$是它长轴的两个端点,且$AQ\perp AP$,$A'Q\perp A'P$,则$Q$点的轨迹方程是$\frac{x^2}{a^2} + \frac{b^2y^2}{a^4} = 1$. 

\item 椭圆的一条直径(过中心的弦)的长,为通过一个焦点且与此直径平行的弦长和长轴之长的比例中项. 

\item 设椭圆$\frac{x^2}{a^2} + \frac{y^2}{b^2} = 1 \left(a > b > 0\right)$的长轴的端点为$A$、$A'$,$P(x_1,y_1)$是椭圆上的点,过$P$作斜率为$-\frac{b^2x_1}{a^2y_1}$的直线$l$,过$A$、$A'$分别作垂直于长轴的直线交$l$于$M$、$M'$,则

(1)$\left|AM\right|\cdot\left|A'M'\right|=b^2$;

(2)四边形$MAA'M'$面积的最小值是$2ab$. 

\item 已知椭圆$\frac{x^2}{a^2} + \frac{y^2}{b^2} = 1 \left(a > b > 0\right)$的右准线$l$与$x$轴相交于点$E$,过椭圆右焦点$F$的直线与椭圆... ...,点$C$在右准线$l$上,且$BC//x$轴,则直线$AC$经过线段$EF$的中点. 

\item $OA$、$OB$是椭圆$\frac{(x-a)^2}{a^2} + \frac{y^2}{b^2} = 1 \left(a >0, b > 0\right)$的两条互相垂直的弦,$O$为坐标原点,则

(1)直线$AB$必经过一个固定点$(\frac{2ab^2}{a^2+b^2},0)$;

(2)以$OA$、$OB$为直径的两圆的另一个交点$Q$的轨迹方程是
$\left(x-\frac{ab^2}{a^2+b^2}\right)^2 + y^2 = \left(\frac{ab^2}{a^2+b^2}\right)^2\left(x\neq 0\right)$. 

\item $P(m,n)$是椭圆$\frac{(x-a)^2}{a^2} + \frac{y^2}{b^2} = 1 \left(a >0, b > 0\right)$上的一个定点,$PA$、$PB$是两条互相垂直的弦,则

(1)直线$AB$必经过一个固定点$(\frac{2ab^2+m\left(a^2-b^2\right)}{a^2+b^2},\frac{n\left(b^2-a^2\right)}{a^2+b^2})$;

(2)以$PA$、$PB$为直径的两圆的另一个交点$Q$的轨迹方程是
$$\left(x - \frac{ab^2+a^2m}{a^2+b^2}\right)^2 + \left(y-\frac{b^2n}{a^2+b^2}\right)^2 = \frac{a^2\left[b^4 +n^2\left(a^2-b^2\right)\right]}{\left(a^2+b^2\right)^2},x\neq m, y\neq n.$$ 

\item 如果一个椭圆短半轴长为$b$,焦点$F_1$、$F_2$到直线$L$的距离分别为$d_1$、$d_2$,那么:

(1)若$d_1d_2=b^2$,且$F_1$、$F_2$在$L$同侧$\Leftrightarrow$直线$L$和椭圆相切;

(2)若$d_1d_2>b^2$,且$F_1$、$F_2$在$L$同侧$\Leftrightarrow$直线$L$和椭圆相离;

(3)若$d_1d_2<b^2$,或$F_1$、$F_2$在$L$异侧$\Leftrightarrow$直线$L$和椭圆相交. 

\item $AB$是椭圆$\frac{x^2}{a^2} + \frac{y^2}{b^2} = 1 \left(a > b > 0\right)$的长轴,$N$是椭圆上的动点,过$N$的切线与过$A$、$B$的切线交于$C$、$D$两点,则梯形$ABCD$的对角线的交点$M$的轨迹方程是$\frac{x^2}{a^2} + \frac{4y^2}{b^2} = 1\left(y\neq 0\right)$. 

\item 设点$P(x_0,y_0)$为椭圆$\frac{x^2}{a^2} + \frac{y^2}{b^2} = 1 \left(a > b > 0\right)$的内部一定点,$AB$是椭圆$\frac{x^2}{a^2} + \frac{y^2}{b^2} = 1$过定点$P(x_0,y_0)$的任一弦,当弦$AB$平行(或重合)于椭圆长轴所在直线时,有$$\left(\left|PA \right|\cdot\left|PB\right|\right)_{min} = \frac{a^2b^2 - \left(a^2y_0^2 + b^2x_0^2\right)}{a^2}. $$

\item 椭圆焦点三角形中,以焦半径为直径的圆必与以椭圆长轴为直径的圆相内切. 

\item 椭圆焦点三角形的旁切圆必切长轴于非焦顶点同侧的长轴端点. 

\item 椭圆两焦点到椭圆焦点三角形旁切圆的切线长为定值$a+c$与$a-c$. 

\item 椭圆焦点三角形的非焦顶点到其内切圆的切线长为定值$a-c$. 

\item 椭圆焦点三角形中,内点到一焦点的距离与以该焦点为端点的焦半径之比为常数$e$(离心率). (注:在椭圆焦点三角形中,非焦顶点的内外角平分线与长轴交点分别称为内、外点). 

\item 椭圆焦点三角形中,内心将内点与非焦顶点连线段分成定比$e$. 

\item 椭圆焦点三角形中,半焦距必为内、外点到椭圆中心的比例中项. 

\item 椭圆焦点三角形中,椭圆中心到内点的距离、内点到同侧焦点的距离、半焦距及外点到同侧焦点的距离成比例. 

\item 椭圆焦点三角形中,半焦距、外点与椭圆中心连线段、内点与同焦点连线段、外点与同侧焦点连线段成比例. 

\item 椭圆焦点三角形中,过任一焦点向非焦顶点的外角平分线引垂线,则椭圆中心与垂足连线必与另一焦半径所在直线平行. 

\item 椭圆焦点三角形中,过任一点向非焦顶点的外角平分线引垂线,则椭圆中心与垂足的距离为椭圆长半轴的长. 

\item 椭圆焦点三角形中,过任一焦点向非焦顶点的外角平分线引垂线,垂足就是垂足同侧焦半径为直径的圆和椭圆长轴为直径的圆的切点. 

\item 椭圆焦点三角形中,非焦顶点的外角平分线与焦半径、长轴所在直线的夹角的余弦的比为定值$e$. 

\item 椭圆焦点三角形中,非焦顶点的法线即为该顶角的内角平分线. 

\item 椭圆焦点三角形中,非焦顶点的切线即为该顶角的外角平分线. 

\item 椭圆焦点三角形中,过非焦顶点的切线与椭圆长轴两端点处的切线相交,则以两交点为直径的圆必过两焦点. 

\item 已知椭圆$\frac{x^2}{a^2} + \frac{y^2}{b^2} = 1$($a>0,b>0$)(包括圆在内)上有一点$P$,过点$P$分别作直线$y=\frac{b}{a}x$及$y=-\frac{b}{a}x$的平行线,与$x$轴分别交于$M$,$N$,交$y$轴于$R$、$Q$,$O$为原点,则:

(1)$\left|OM\right|^2 + \left|ON\right|^2 = 2a^2$; 

(2)$\left|OQ\right|^2 + \left|OR\right|^2 = 2b^2$. 

\item 过平面上的$P$点作直线$l_1:y=\frac{b}{a}x$及$l_2:y=-\frac{b}{a}x$的平行线,分别交$x$轴于$M$、$N$两点,交$y$轴于$R$、$Q$,

(1)若$\left|OM\right|^2 + \left|ON\right|^2 = 2a^2$,则$P$的轨迹方程是$\frac{x^2}{a^2} + \frac{y^2}{b^2} = 1 \left(a > 0, b > 0\right)$;

(2)若$\left|OQ\right|^2 + \left|OR\right|^2 = 2b^2$,则$P$的轨迹方程是$\frac{x^2}{a^2} + \frac{y^2}{b^2} = 1 \left(a > 0, b > 0\right)$.

\item 点$P$为椭圆$\frac{x^2}{a^2} + \frac{y^2}{b^2} = 1 \left(a > 0, b > 0\right)$(包括圆在内)在第一象限的弧上任意一点,过$P$引$x$轴、$y$轴的平行线,交$x$轴、$y$轴于$N$、$M$,交直线$y=-\frac{b}{a}x$于$Q$,$R$,记$\triangle OMQ$与$\triangle ONR$的面积为$S_1$、$S_2$,则
$S_1 + S_2 = \frac{ab}{2}. $

\item 点$P$为第一象限内的一点,过$P$引$x$轴、$y$轴的平行线,交$x$轴、$y$轴于$N$、$M$,交直线$y=-\frac{b}{a}x$于$Q$、$R$,记$\triangle 0MQ$与$\triangle 0NR$的面积为$S_1$、$S_2$,已知$S_1 + S_2 = \frac{ab}{2}$,则点$P$的轨迹方程是$$\frac{x^2}{a^2} + \frac{y^2}{b^2} = 1 \left(a > 0, b > 0\right). $$

\end{enumerate}


\end{document}
